\chapter{Introduzione}
La presente relazione riguarda il progetto e la verifica strutturale di alcuni elementi componenti un edificio composto da 4 piani di cui 3 fuori terra sito in Provincia di Trento ad un'altitudine di \SI{300}{\meter} sopra il livello del mare.
Le dimensioni in pianta sono di circa $28 \times \SI{42}{\metre}$ per una altezza totale di \SI{12.45}{\meter} di cui \SI{9.70}{\meter} fuori terra. 
La cubatura totate è perciò di circa \SI{14650}{\metre\cubed}.

\paragraph{Composizione strutturale dell'edificio}
La struttura dell'edificio è a telaio in calcestruzzo armato, da progetto preliminare caratterizzata da pilastri quadrati di sezione \SI{30}{\centi\meter} e da travi sia a spessore che non, rispettivamente larghe \SI{60}{\centi\meter} ed alte come lo spessore del solaio, e $30 \times \SI{50}{\centi\metre}$.

I solai strutturali sono di tipo a lastre tralicciate Predalle tra il piano interrato e il piano terra e nel solaio di copertura. 
Il peso ultimato di tale solaio di \SI{3.6}{\kilo\newton\per\square\meter}.
I solai tra piano terra e piano primo e tra piano primo e piano secondo sono di tipo a travetti e latero-cemento.
Il peso ultimato di tale solaio di \SI{3.20}{\kilo\newton\per\square\meter}.

I solai dei piano intermedi sono costituiti da un pacchetto con calcestruzzo alleggerito, massetto di allettamento, intonaco e pavimento in ceramica.
Il solaio nella zona del terrazzo è costituito da isolante, impermealizzazione, massetto in calcestruzzo e pavimento.
Il solaio di copertura è costituito da isolante, massetto in calcestruzzo alleggerito, impermealizzazione e ghiaia.

Le pareti divisorie interne sono costituite da muratura in laterizio e intonaco in ambo i lati.
Le pareti perimetrali sono costituite da muratura in laterizio, cappotto esterno e intonaco.

Il piano terra è adibito a negozi, il piano primo ad uffici aperti al pubblico, il piano secondo a civile abitazione.
Il piano interrato è adibito a garage.

\paragraph{Oggetto della relazione}
Gli elementi che si andranno ad analizzare saranno una trave ed un solaio del primo piano e il pilastro P27. 
Tali elementi sono ben evidenziati nelle piante a seguire insieme alle sollecitazioni in essi agenti.

Subito dopo vengono elencati normative e materiali impiegati nell'analisi strutturale.

Seguono poi i capitoli di calcolo relativi agli elementi in esame.
La trave verrà analizzata per sollecitazioni agli SLU e SLE, solaio e pilastro per quelle agli SLU.

In allegato al presente documento sono riportati degli elaborati grafici eseguiti in software CAD.